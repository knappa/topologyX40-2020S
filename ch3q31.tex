\setcounter{question}{30} % one less than question number
\begin{question}[Ewing]
If $f\colon X\to Y$ is a function from set $X$ into a pseudometric space $(Y,d)$, then $\rho\colon X\times X\to \R$ defined by $\rho(x_1,x_2)=d(f(x_1),f(x_2))$ is a pseudometric for $X$ and the topology generated by $\rho$ is the weak topology by $f$.
\end{question}

\begin{proof}
First, we show that $\rho$ is a pseudometric for $X$. Given arbitrary $x_1,x_2,x_3\in X$, we have $f(x_1), f(x_2), f(x_3)\in Y$. Therefore, since $d$ is a pseudometric for $Y$, we also have
\begin{enumerate}
\item $\rho(x_1,x_2)=d(f(x_1),f(x_2))\ge 0$ for all $(x_1,x_2)\in X\times X$
\item $\rho(x_1,x_1)=d(f(x_1),f(x_1))=0$ for all $x_1\in X$
\item $\rho(x_1,x_2)=d(f(x_1),f(x_2))=d(f(x_2),f(x_1))=\rho(x_2,x_1)$ for all $(x_1,x_2)\in X\times X$, and
\item $\rho(x_1,x_3)=d(f(x_1),f(x_3))\le d(f(x_1),f(x_2))+d(f(x_2),f(x_3))=\rho(x_1,x_2)+\rho(x_2,x_3)$ for all $x_1,x_2,x_3\in X$.
\end{enumerate}
Therefore $\rho$ is a pseudometric for $X$.

Second, we show that $T_{\rho}$, the topology generated by $\rho$, is the weak topology by $f$. Since the topology generated by a pseudometric is the union of all open cells defined by the pseudometric, for any $p,x\in X, r\in\R$ we have $$T_{\rho}=\bigcup\{C_x(p;r)\colon \rho(p,x)<r\}=\bigcup\{C_x(p;r)\colon d(f(p),f(x))<r\}.$$ Similarly, since $d$ is a pseudometric for $Y$, it is a pseudometric for subspace $(f(Y),d)$ with the subspace topology $$T_d=\bigcup\{C_y(f(p);r)\colon d(f(p),f(x))<r; f(p),f(x)\in Y; r\in\R\}.$$ Clearly, whenever $f(x)\in C_y(f(p);r)$ we have $x\in C_x(p;r)$. Hence for any $f(p)\in f(Y)$, $$f^{-1}(C_y(f(p);r))=C_x(p;r)$$ and consequently $$T_{\rho}=\bigcup\{C_x(p;r)\}=\bigcup\{f^{-1}(C_y(f(p);r))\}=f^{-1}\Bigl(\bigcup\{(C_y(f(p);r))\}\}\Bigr)=f^{-1}(T_{d})$$ which is the weak topology by $f$ for $X$.
\end{proof}

%%% Local Variables:
%%% mode: latex
%%% TeX-master: "main"
%%% End:
