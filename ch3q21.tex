\setcounter{question}{20} % one less than question number
\begin{question}[Ewing]
If $X=\R$ and $S,T$ the topologies generated by bases $\{[a,b)\colon a,b\in X, a<b\}$ and $\{(a,b]\colon a,b\in X, a<b\}$, respectively, is the identity function $i\colon (X,S)\to(X,T)$ a homeomorphism and are the two topological spaces homeomorphic?
\end{question}

(a) The identity function is not a homeomorphism.

\begin{proof}
To be a homeomorphism, $i$ must be continuous and, hence, for any open set $U\in X$ the set $f(U)\in Y$ must also be open. The interval $I=[0,1)$ is open in $(X,S)$, but $f(I)=[0,1)$ is not open in $(X,T)$ because it cannot be constructed as a union of basis elements $\{(a,b]\colon a,b\in X, a<b\}$, since any such union must include its supremum, which $I$ does not.
\end{proof}

(b) The two topological spaces are homeomorphic.

\begin{proof}
Let $f(x)=-x$. This function is one-to-one. For any $a,b,x\in X=\R$ when $a\le x<b$ we have $-b<-x\le -a$. Therefore for any element $[a,b)$ in the base of $S$ (i.e., with $a<b$), $f\left([a,b)\right)=(-b,-a]$ with $-b<-a$, which is in the base of $T$. Similarly, for any element $(a,b]$ in the base of $T$, $f^{-1}\left((a,b]\right)=[-b,-a)$ is in the base of $S$. Since all open sets in $S$ or $T$ must be unions of elements of the respective base, $f$ and $f^{-1}$ both take open sets to open sets and so are continuous. Hence $f$ is a homeomorphism, making $(X,S)\equiv(X,T)$.
\end{proof}
%%% Local Variables:
%%% mode: latex
%%% TeX-master: "main"
%%% End:
